
When inspecting residuals there is a danger of hypothesis fishing and if multiple true hypotheses are tested it is likely that some of them will be rejected. Therefore it is valuable to reserve part of the data for validation, so that a pattern’s significance is not tested on the same data set which suggested the pattern \citep{thygesen2017validation}.

%To conduct a hindcast involves fitting a model using a tail-cutting procedure, where data were deleted from year 2018 through to 1999 sequentially and then the data from year 1950 to the last year in the cut data set were used to make predictions of what will happen inhttps://www.overleaf.com/project/5ca9fed01e2a625dbaeae31f thehttps://www.overleaf.com/project/5ca9fed01e2a625dbaeae31f years that were removed.

To conduct a hindcast involves fitting a model using a tailcutting procedure, where data are deleted sequentially for $n$ years, i.e. from the last year $y$ through to $y−n$ and then the data from year 1 to $y−n−1$ are used to make predictions of what will happen in years $y−n$ to $y$.

\begin{algorithm}[!ht]
\begin{algorithmic}[1]

\State Conduct assessment for all years, i.e. $t=2018$
\For ~ $t$ = 2018 to 1999
\State Remove all $U_{i,t}$ for $t$ to 2018
\State Run assessment
\State Estimate $\hat{U}_{i,t}$ 
\EndFor
\caption{Hindcast~\citep{kell2016xval}}
\label{Hindcast}
\end{algorithmic}
\end{algorithm}

When conducting a hindcast, as when conditioning stock assessment models, it is assumed that modelled variables are observable, processes exhibit constancy of structure in time, including those not specified in the model, and that collection of accurate and sufficient data is possible \citep{hodges1992you}.

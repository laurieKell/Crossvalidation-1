
Analysis of residuals is a common way to determine a model’s goodness-of-fit \citep{Cox1968general}, since  non-random patterns in the residuals may indicate model misspecification, serial correlation in sampling/observation error, or heteroscedasticity. 

When inspecting residuals there is a danger of hypothesis fishing and if multiple true hypotheses are tested it is likely that some of them will be rejected. Therefore it is valuable to reserve part of the data for validation, so that a pattern’s significance is not tested on the same data set which suggested the pattern \citep{thygesen2017validation}. We therefore conduct a model-free hindcast

When conducting a hindcast it is assumed that modelled variables are observable, processes exhibit constancy of structure in time, including those not specified in the model, and that collection of accurate and sufficient data is possible \citep{hodges1992you}.


%To conduct a hindcast involves fitting a model using a tail-cutting procedure, where data were deleted from year 2018 through to 1999 sequentially and then the data from year 1950 to the last year in the cut data set were used to make predictions of what will happen inhttps://www.overleaf.com/project/5ca9fed01e2a625dbaeae31f thehttps://www.overleaf.com/project/5ca9fed01e2a625dbaeae31f years that were removed.

%A problem with retrospective analysis is that model outputs are used and bias is measure%d using RE a measure of variance not bias. For example what if the estimates from the full time series are biased by shrinking estimates towards recent historical values, or by over-fitting to conflicting data?  Furthermore in statistics shrinkage is used to reduce variance at the expense of increasing bias. 

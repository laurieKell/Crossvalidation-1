
\subsection{Hindcast}

Retrospective analysis  \parencite{hurtado2014looking} is commonly used to evaluate the stability of stock assessment estimates from alternative models. Observations are sequentially removed from the terminal year, the model is then refitted to the truncated series and the difference between between estimates from the full and truncated time-series compared using the relative error (RE, a measure of bias). 

Hindcasting like traditional retrospective analysis involves fitting a model using a tailcutting procedure, where data are deleted sequentially for $n$ years, i.e. from the last year $T$ through to $T−n$, the additional step in the hindcast is that then the data from year 1 to $T - n - 1$ are used to make predictions of what will happen in years $T - n$ to $T$.

Since assessment cycles are typically for three years  with advice \parencite{fricker2013three} we projected the truncated estimates for 3 years. We chose 3 years as that is essentially the time-step between assessments in most tuna Regional Fisheries Management Organisations.

\begin{algorithm}[!ht]
\begin{algorithmic}[1]
\State Fit model up to and including terminal year $T$
\For{$t$ = $T - 3$ to n} 
\State fit model to data up to time $t$ \For {$i$ = $t$ to $t + 3$} 
\State Estimate $\hat{y_i}$
\EndFor
\EndFor
\caption{Hindcast}
\label{Hindcast}
\end{algorithmic}
\end{algorithm}

When conducting the hindcast it is assumed that modelled variables are observable, processes exhibit constancy of structure in time, including those not specified in the model, and that collection of accurate and sufficient data is possible \parencite{hodges1992you}.

\subsubsection{Prediction Skill}

The use of model based quantities means that bias can not actually be quantified. For example a reduction in both relative error (a measure of bias) and mean squared error (a measure of variance) can be achieved by shrinking terminal estimates towards recent historical values, at the expense of prediction skill. The absence of retrospective patterns in model based quantities, therefore, while reassuring is not sufficient for model validation, and model-free validation using prediction residuals should be used as well.

Inspection of residuals is a common way to determine a model’s goodness-of-fit \parencite{cox1968general}, since  non-random patterns in the residuals may indicate model misspecification, serial correlation in sampling/observation error, or heteroscedasticity. When inspecting residuals, however, there is a danger of hypothesis fishing/testing?, i.e. [I read the sentence below a few times but could not make sense of it probably need revising?] choosing a scenario retrospectively retrospect, while if multiple true hypotheses are tested it is likely that some of them will be rejected. Therefore it is valuable to reserve part of the data to test for validation, so that a pattern’s significance is not tested on the same data set which suggested the pattern \parencite{thygesen2017validation}. For this reason we conduct a model-free hindcast to estimate prediction skill.

\subsubsection{Metrics}

For the evaluation of model-based quantities, we use Mohn's rho ($\rho_M$ \toshi{$\rho_{Mr}$?}, equation \ref{eqn:mohn}) and a modified version ($\rho_{Mr}$ \toshi{$\rho_{p}$?}, equation \ref{eqn:mohn2}). 
\toshi{IF my understanding above on $\rho_{p}$ is correct, this is also for assessing prediction skill but "model based". I think we can highlight here difference in model-based and model-free in addition to model validation with/without prediction.}

We used the mean absolute scaled error (MASE, equation \ref{eqn:mase}) to evaluate prediction skill. The best statistical measure to use depends, however, on the objectives of the analysis and using more than one measure can be helpful in providing insight into the nature of observation and process error structures \parencite{kell2016xval}. Therefore we also use root mean squared error ($E^{\prime 2}$, equation \ref{eqn:rmse}), correlation ($\rho$) and the standard deviation ($\sigma$). 

$E^\prime$ is a commonly used metric, as the square root of a variance it can also be interpreted as the standard deviation of the unexplained variance,  lower values indicate better fits. $E^\prime$ is sensitive to outliers, however,and favours forecasts that avoid large deviations from the mean and cannot be used to compare across series. The correlation ($\rho$) in contrast is unaffected by the amplitude of the variations, insensitive to biases and errors in variance, and can be used to compare across series. $E^{\prime 2}$ and $\rho$ are related by the cosine rule i.e.

 \begin{equation} 
 E^{\prime 2} = \sigma_o^2 + \sigma_f^2 - 2\sigma_o\sigma_f\rho
 \end{equation}

Where the reference set ($o$) are the observations not included in the retrospective assessment and the values ($f$) are their estimates. 
 
This means that $E^\prime$, $\rho$ and $\sigma_f$ can be summarised simultaneously in a single diagram\parencite{taylor2001summarizing} providing a concise statistical summary of how well patterns match each other and are therefore especially useful for evaluating multiple aspects or in gauging the relative skill of different models \parencite{griggs2002climate}.


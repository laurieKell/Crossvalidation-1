\section{Material and Methods}
Indices of abundance are a key contributor to the overall likelihood when fitting stock assessment models to data \parencite{whitten2013accounting}. The Sum of Squared Errors (SSE) between observed and predicted indices in log-space is the measure of fitness. When comparing models, however, the SSE is problematic because complex models tend to have many parameters to allow flexibility when fitting, which may result in a low SSE due to overfitting. Therefore, information criteria, such as AIC, have been developed to aid in model selection. AIC is only a relative measure of the appropriateness of models, however, and additional diagnostic tests are required for model validation. This is of particular importance for stock assessment models where only a single historical data set exists, and the system can not be observed directly. 

Therefore to compare the different family of models, we extend retrospective analysis to conduct a model-based and model-free hindcast, by adding the additional step of projecting over the truncated years. The objective is to test the model based on past events by comparing outputs with known observations.  Conducting a model-free hindcast \parencite{kell2016xval} allows prediction skill to be estimated, defined as any measure of the accuracy of a forecasted value compared to the actual (i.e. observed) value that is not known by the model \parencite{glickman2000glossary}. 

\subsection{Materials}

The data are those used in the stock assessment of yellowfin tuna (\textit{Thunnus albacares}) by the Indian Ocean Tuna Commission (IOTC). These include time series of total catch and four catch-per-unit-effort (CPUE) indices based on various long-line fisheries, spatially stratified in four regions (figure \ref{fig:map}). Yellowfin tuna supports one of the largest tuna fisheries in the Indian Ocean, with catches currently exceeding 400,000t. They are harvested by a variety of gears, from small-scale artisanal fisheries, to large gillnetters, and industrial longliners and purse seiners \parencite{fiorellato2019tt}.

The assessment is conducted using Stock Synthesis \parencite[SS3,][]{methot2013stock} which implements an age and spatially structured model that reflects the complex population and fishery dynamics of the stock. Model development has focused on spatial structure to account for the differences in regional exploitation patterns, incorporating seasonal movement dynamics, resolving data conflicts, and exploring non-stationary in selectivity and catchability \parencite{urtizberea2018yft}.

The most recent assessment established a base case as a reference model for diagnostics along with scenarios to capture a range of uncertainties \parencite{fu2018yft}. The assessment indicates that the stock  has declined substantially since 2012, and spawning stock biomass in 2017 is now estimated to be close to the historical lowest level. The stock is estimated to be overfished, and the IOTC has implemented a rebuilding plan to reduce the overall fishing pressure  

The base case is spatially disaggregated into two tropical regions that encompass the main year-round fisheries and two austral, subtropical regions where the long-line fisheries occur more seasonally \parencite{langley2015yft}, with reciprocal movement assumed to occur between adjacent regions (Figure \ref{fig:map}). The model is based on a quarterly time step to approximate the continuous recruitment and rapid growth seen in the stock. Twenty-five fisheries were defined based on fishing gear, region, time period, fishing mode and vessel type. Most fisheries were modelled allowing flexibility in selectivity (e.g. cubic spline or double normal), whereas long-line selectivity was constrained to be fully selective for the older ages.  The population comprised 28 quarterly age-classes with an assumed unexploited equilibrium initial state in each region. 

Recruitment occurs in the two equatorial regions with temporal deviates in the regional distribution and was assumed to follow a Beverton and Holt stock recruitment relationship (with a steepness of 0.8 and recruitment standard deviation of 0.6).  Growth was parameterised using age-specific deviates on the $k$ growth parameter to mimic the non-von Bertalanffy growth of juvenile and the near linear growth of adults. Natural mortality is variable with age, with the relative trend in age-specific natural mortality based on the values applied in the Pacific Ocean [Maunder & Aires-da-Silva 2012]. 

[Maunder, M.N., Aires-da-Silva, A. 2012. A review and evaluation of natural mortality for the assessment and management of yellowfin tuna in the eastern Pacific Ocean. External review of IATTC yellowfin tuna assessment. La Jolla, California. 15-19 October 2012. Document YFT-01-07.]


The data used for fitting are catch and length composition data, long-line CPUE indices, tagging recaptures, and environmental data. The length composition was weighted such that they were sufficient to provide reasonable estimates of fishery selectivity and recruitment trends but not directly influence the trends in stock abundance. Regional environmental indices (current and sea temperature) allows seasonal and temporal variations to be incorporated in the estimation of fish movement. 

The CPUE indices represent the primary source of information on abundance and is based on a composite long-line index from the main distant water fleets \parencite{hoyle2018yft}. Indices in each region were standardised using generalised linear models that accounted for differences in targeting practices and catchability amongst fleets, based on gear configurations and species composition. The reason for this is because tuna long-line fishing strategies have changed over time \parencite{hampton2005}. In the assessment, the CPUE indices across regions were linked by a common catchability coefficient, thus improving the ability of the model to estimate the distribution of biomass by region \parencite{langley2015yft, fu2018yft}.

Tag release/recovery data collected from the main phase of the Indian Ocean large-scale tuna tagging programme [Hallier 2008] were integrated into the model to inform estimates of fishing mortality, abundance, and movement. 

[Hallier 2008. STATUS OF THE INDIAN OCEAN TUNA TAGGING PROGRAMME - RTTP-IO. IOTC-2008-WPTDA-10]

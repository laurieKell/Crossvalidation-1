Yellowfin tuna (Thunnus albacares) supports one of the largest tuna fisheries in the Indian Ocean, with catches currently exceeding 400,000t, where they are harvested by a variety of gears, from small-scale artisanal fisheries, to large gillnetters, and industrial longliners and purse seiners \citep{fiorellato2019tt}

The stock is estimated to be overfished due to an increase in catch levels in recent years and consequently, IOTC introduced a rebuilding plan in 2016 to reduce the overall fishing pressure on the stock. Since 2015, the assessment of has been implemented using Stock Synthesis \citep[SS3][]{methot2013stock}. The SS3 assessment \citep[e.g.][]{\cite{urtizberea2018yft} implements an age and spatially structured model that reflected the complex population and fishery dynamics of the stock. Model development has focused primarily on determining an appropriate spatial structure that accounts for the differences in regional exploitation pattern and the level of tag mixing, incorporating seasonal movement dynamics, resolving data conflicts, and exploring non-stationary in selectivity and catchability.  

The most recent assessment \citep{fu2018yft} established a base case as a reference model for diagnostics along with scenarios that capture a range of uncertainties. The assessment indicated that stock  has declined substantially since 2012, and spawning stock biomass in 2017 is now estimated to be close to the historical lowest level. 

The base case is spatially dis-aggregated into two tropical regions that encompass the main year-round fisheries and two austral, subtropical regions where the longline fisheries occur more seasonally \citep{Langley2015}, with reciprocal movement assumed to occur between adjacent regions (Figure \ref{fig:map}). The model is based on a quarterly time step to approximate  continuous recruitment and rapid growth seen in the stock. Twenty-five fisheries were defined based on fishing gear, region, time period, fishing mode and vessel type. Most fisheries were modelled allowing flexiblity in selectivity (e.g. cubic spline or double normal), whereas long-line selectivity was constrained to be fully selective for the older fish.  The population comprised 28 quarterly age-classes with an assumed unexploited equilibrium initial state in each region. 

Recruitment occurs in the two equatorial regions with temporal deviates in the regional distribution and was assumed to follow a Beverton and Holt stock recruitment relationship (with a steepness of 0.8 and recruitment standard deviation of 0.6). 

Growth was parameterised using age-specific deviates on the k growth parameter to mimic the non-von Bertalanffy growth of juvenile and near linear growth of adults. 

Natural mortality is variable with age with the relative trend in age-specific natural mortality based on the values applied in the Pacific Ocean. 

The data used for fitting are catch and length composition data, longline CPUE indices, tagging recaptures, and environmental data. The length composition was weighted such that they were sufficient to provide reasonable estimates of fishery selectivity and recruitment trends but not directly influence the trends in stock abundance. Regional environmental indices (current and sea temperature) allows seasonal and temporal variations to be incorporated in the estimation of the fish movement. 

Tag release/recovery data collected from the main phase of the IO large-scale tuna tagging programme were integrated into the model to inform estimates of fishing mortality, abundance, and movement. 

The CPUE indices represent the primary source of information on abundance and is based on a composite longline index from the main distant water fleets \citep{Hoyle2018a}.
Indices in each region were standardised using generalized linear models that accounted for differences in targeting practices and catchablity amongst fleets, based on gear configurations and species composition. The reason for this is because tuna longline fishing strategies have changed over time \citep{Hampton2005}. In the assessment, the CPUE indices across regions were linked by a common catchability coefficient, thus improving the ability of the model to estimate the distribution of biomass by region \citep{Langley2015, Fu2018}. 

%%%%%%%%%%%%%%%%%%%%%%%%%%%%%%%%%%%%%%%%%%%%%%%%%%%%%%%%%%%%%%%%%%%%%%%%%%%%%%%%%%%%%%%%%%%%
%% Ignored after here %%%%%%%%%%%%%%%%%%%%%%%%%%%%%%%%%%%%%%%%%%%%%%%%%%%%%%%%%%%%%%%%%%%%%%
%%%%%%%%%%%%%%%%%%%%%%%%%%%%%%%%%%%%%%%%%%%%%%%%%%%%%%%%%%%%%%%%%%%%%%%%%%%%%%%%%%%%%%%%%%%%

\iffalse

Yellowfin tuna (Thunnus albacares) support one of the largest tuna fisheries in the Indian Ocean, where they are harvested with a variety of gear types, from small-scale artisanal fisheries, to large gillnetters, and industrial longliners and purse seiners, with current total catches over 400,000t \cite{fiorellato2019tt}

The stock is estimated to be overfished due to an increase in catch levels in recent years and consequently, IOTC introduced a rebuilding plan in 2016 to reduce the overall fishing pressure on the stock. Since 2015, the assessment of IO yellowfin tuna was implemented using the Stock Synthesis software (SS3) (Langley 2015, 2016, Fu et al. 2018, Urtizberea et ta. 2019). The SS3 assessment implements an age- and spatially structured model that reflected the complex population and fishery dynamics of the species. To date model development has focused on determining an appropriate spatial structure that accounts for the differences in regional exploitation pattern and the level of tag mixing, incorporating seasonal movement dynamics, resolving data conflicts, and exploring non-stationary processes (e.g. selectivity and catchability).  The most recent benchmark assessment \cite{fu2018yft} has established a model ensemble to capture a range of uncertainties and adopted a reference case for diagnostics purposes (IOTC 2018). The assessment indicated stock biomass declined substantially since 2012, with the spawning biomass in 2017 estimated to be close to the historically low level. We use the reference model as the test case in the paper which is briefly summarized as below.  More details of the model specification are in in \cite{fu2018yft}.

The reference model is spatially disaggregated into two tropical regions that encompass the main year-round fisheries and two austral, subtropical regions where the longline fisheries occur more seasonally (Langley 2015), with reciprocal movement assumed to occur between adjacent regions (Figure 1).  The time period (1950 – 2017) were compiled into quarters (defined to be model “years”) to approximate the continuous recruitment and rapid growth. Twenty-five fisheries were defined based on fishing gear, region, time period, fishing mode and vessel type. Most fisheries used a flexible selectivity form (e.g. cubic spline or double normal) whereas the longline selectivity was constrained to be fully selective for the older fish.  The population is comprised of 28 quarterly age-classes with an unexploited, equilibrium initial state in each region. Recruitment occurs in the two equatorial regions with temporal deviates in the regional distribution and was assumed to follow a BH stock recruitment relationship (steepness of 0.8 and sigmaR of 0.6). The growth was parameterised using age-specific deviates on the k growth parameter to mimic the non-von Bertalanffy growth of juvenile and near linear growth of adults. Natural mortality is variable with age with the relative trend in age-specific natural mortality based on the values applied in the Pacific Ocean.  The data fitted in the model consist of catch and length composition data, longline CPUE indices, tagging recaptures, and environmental data. The length composition was weighted such that they were sufficient to provide reasonable estimates of fishery selectivity and recruitment trends but not directly influence the trends in stock abundance. Regional environmental indices (current and sea temperature) allows seasonal and temporal variations to be incorporated in the estimation of the fish movement. Tag release/recovery data collected from the main phase of the IO large-scale tuna tagging programme were integrated into the model to inform estimates of fishing mortality, abundance, and movement. 

The CPUE indices represent the primary source of information on abundance and was based on a composite longline catch effort data from main distant water longline fleets standarised in a unified framework (Hoyle et al. 2018a). The Joint standardization utilized substantive logbook programmes with extensive history and broad sampling coverage and is considered a better approach than using fleet-specific, potentially conflicting indices. A data filter was used to indenfity credible data subsets to mininimze the conficts of trend amongst fleets which are thought to be mainly caused by lower spatial coverage or missiporting from fleets (Hoyle et al. 2018b). Indices in each region were standardised using generalized linear models that accounted for differences in targeting practices and catchablity amongst fleets, based on gear configurations and species composisiton (in areas with known distinctive targeting behaviour). This is important as tuna longline in all oceans is known to have changed fishing strategies over time (Hampton et al. 2005). In the assessment model, the respective CPUE indices among regions were linked by a common catchability coefficient, thus improving the ability of the model to estimate the distribution of biomass among regions (Langley 2015, Fu et al. 2018). As such the regional indices were scaled by the respective regional scaling factor which incorporated both the size of the region and the catch rate to index the relative level of exploitable long-line biomass among region (Hoyle and Langley 2020). Temporal trends in the standardized indices vary within regions, with greater decline in CPUE in tropical areas close to the equator (Figure 2). The trend does not appear to be consistent with the general perception of the high degree of mixing of the populations between regions 1 and 2 as induced by oceanographic conditions but may have been associated with greater depletion of areas subject to more purse seine fishing (Hoyle et al. 2018b).  It is also possible that the recent steeper decline of CPUE indices in region 1 could be reflective of fishing operation and/or a change in the fleet composition rather than abundance (Fu et al 2018).  In addition, the exceptionally high peak in CPUE indices the late 1970s is most likely associated with reporting and data management (Hoyle et al 2017), and the spike around 2012 in region 1 may have reflected changes in the population or catchability as vessels returned to areas once affected by piracy. 

\fi


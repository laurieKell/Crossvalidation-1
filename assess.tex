
There has been a recent trend in stock assessment toward the use of integrated analysis that combines several sources of data into a single model by a joint likelihood for the observed data \citep[e.g.][]{doubleday1976least,fournier1982general,maunder2013review}. Datasets include records of catches and landings, indices of abundance based on catch per unit (CPUE) and from research surveys, and length classes and ages compositions based on samples. A commonly used integrated assessment method is Stock Synthesis \citep[SS3,][]{methot2013stock} that can be configured in multiple ways, allowing for a large number of scenarios to be developed that reflect uncertainty.

\cite{Maunder2015} proposed an age-structured production model (ASPM) based on SS as a diagnostic of process dynamics. This evaluates whether the observed catches alone can explain trends in the index of abundance. Since if the ASPM is able to fit the indices of abundance that have good contrast then a production function is likely to exists (i.e. the dynamics are driven by density dependent processes), and the indices provide information about absolute abundance. If  there is not a good fit to the indices, then the catch data alone cannot explain the trends in the indices. This can have several causes: namely (i) stock dynamics are recruitment-driven, (ii) the stock has not yet declined to the point at which catch is a major factor influencing abundance; (iii) the indices of relative abundance are not proportional to abundance; or (iv) the base-case model is incorrectly specified.

In the later case this may indicate that the model structure is misspecified or the data are incorrect. The ASPM has been shown \citep{carvalho2017can} to be the best method for detecting misspecification of the key systems-modeled processes that control the shape of the production function. This is a problem since when conducting an integrated assessments many of the required parameters are difficult to estimate \citep[e.g.][]{lee2011m,lee2012steepness} and have to be fixed or priors used. An alternative is to use biomass dynamic models based on a production function, that requires the estimation and fixing of fewer parameters. An example is JABBA an open source package that presents a unifying, flexible framework for biomass dynamic modelling, runs quickly, and generates reproducible stock status estimates \citep{winker2018jabba}.

The main datasets were those used in the Indian Ocean Yellowfin Tuna Assessment. The main data are time series of total catch and four catch per unit effort (CPUE) indices. Spatial stratification of the Indian Ocean is by four regions (figure \ref{fig:map}). The black arrows represent the configuration of the movement parameterisation.  Density contours represent of the dispersal of tag releases (red) and subsequent recaptures from Indian Ocean Regional tuna tagging program. Green circles represent the distribution of catches from the longline fishery aggregated by 5 longitude * 5 latitude for 1980 – 2017 (max. = 133 770 t).


\iffalse
There has been a recent trend in stock assessment toward the use of integrated analysis that combines several sources of data into a single model by a joint likelihood for the observed data \citep[e.g.][]{doubleday1976least,fournier1982general,maunder2013review}.  Datasets include records of catches and landings, indices of abundance based on catch per unit (CPUE) and from research surveys, and length classes and ages compositions based on samples. 

A commonly used integrated assessment method is Stock Synthesis \citep[SS3,][]{methot2013stock} that can be configured in multiple ways. For example Maunder and Piner (2015) proposed an age-structured production model (ASPM) based on SS as a diagnostic of process dynamics. This diagnostic evaluates whether the observed catches alone (taken out of approximately the correct ages) can explain trends in the index of abundance. On the one hand, Maunder and Piner (2015) suggest that if the ASPM is able to fit well to the indices of abundance that have good contrast (i.e. those that have declining as well as increasing trends), the production function likely exists, and the indices will provide information about absolute abundance. On the other hand, the authors suggest that if there is not a good fit to the indices, then the catch data alone cannot explain the trajectories depicted in the indices of relative abundance. This can have several causes: (i) the stock is recruitment-driven; (ii) the stock has not yet declined to the point at which catch is a major factor influencing abundance; (iii) the base-case model is incorrect; or (iv) the indices of relative abundance are not proportional to abundance. Alternatively, failure in the ASPM may indicate a system that is not well organized (e.g., stock structures or data are incorrect) so that a real fishing signal is lost or where unknown environmental drivers control population abundance. The ASPM was shown via simulation analyses to be the only tested diagnostic capable of detecting misspecification of the key systems-modeled processes that control the shape of the production function \citep{carvalho2017can}.

A problem with integrated assessments is that many of the parameters required are difficult to estimate in practice \citep[e.g.][]{lee2011m,lee2012steepness} and have to be fixed or priors developed. An alternative is to use biomass dynamic models based on a surplus production function, that requires the estimation and fixing of fewer parameters. Once example is JABBA an open source package that presents a unifying, flexible framework for biomass dynamic modelling, runs quickly, and generates reproducible stock status estimates \citep{winker2018jabba}.
\fi


%\cite{maunder2006interpreting}
%\cite{langley2015yft}
%\cite{langely2016yft}
%\cite{langley2016yft2}
%\cite{urtizberea2018yft}



\subsection{Assessment Methods}

There has been a recent trend in stock assessment toward the use of integrated analysis that combines several sources of data into a single model by a joint likelihood for the observed data \parencite[e.g.][]{doubleday1976least,fournier1982general,maunder2013review}. Datasets include records of catches and landings, indices of abundance based on catch per unit (CPUE) or from research surveys, and length and age compositions based on samples. An example of commonly used integrated assessment method is SS3 that can be configured in multiple ways, allowing for a range of scenarios to be developed to reflect uncertainty.

For example \cite{maunder2015contemporary} proposed a deterministic implementation of an age-structured production model (ASPM) as a diagnostic of process dynamics. Selectivity in ASPM is parameterised based on the selectivity estimated by a "full" SS model. The model is then fitted to the abundance indices, assuming a deterministic spawning recruitment relationship, and without the size composition data contributing to the likelihood function. This enables an evaluation of whether the observed catches alone can explain trends in the index of abundance. If the ASPM is able to fit the indices of abundance well then a production function is likely to exist (i.e. the dynamics are driven by density dependent processes), and the indices provide information about absolute abundance. If  the fit is poor, then the catch data alone cannot explain the trends in the indices. This can have several causes, namely (i) stock dynamics are recruitment-driven, (ii) the stock has not yet declined to the point at which catch is a major factor influencing abundance; (iii) the indices of relative abundance are not proportional to abundance;  (iv) the model is incorrectly specified; or (v) the data are incorrect

The ASPM has been shown \parencite{carvalho2017can} to be the best method for detecting misspecification of the key systems-modeled processes that control the shape of the production function. This is a problem since many of the required parameters in integrated assessments are difficult to estimate \parencite[e.g.][]{lee2011m,lee2012steepness} and have to be fixed or priors used. 

An alternative to an integrated assessment is to use a biomass dynamic model, based on an explicit production function, that requires the estimation and fixing of fewer parameters. An example is JABBA, an open source package that presents a unifying, flexible framework for biomass dynamic modelling, runs quickly, and generates reproducible stock status estimates \parencite{winker2018jabba}.The model uses a Pella Tomlinson production function that allows the shape of the production function to be varied, and alternative assumptions about productivity, stock status and reference points to be evaluated. 

The base case SS assessment conducted by the Indian Ocean Tuna Commission (IOTC) for Yellowfin Tuna, was reconfigured as an ASPM and as a biomass dynamic assessment. In the later case in order to mimic the dynamics of the base case assessment, the production function parameters were tuned to the base case Stock synthesis assessment. 
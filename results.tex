\section{Results}

The retrospective analysis and the three step ahead predictions are shown in figure \ref{fig:retro}. These show the estimates of SSB and F relative to their maximum sustainable yield ($MSY$) reference points  $B_{MSY}$ and $F_{MSY}$. The retrospective analysis shows that SS base case and JABBA assessments are biased, and bias increases for the projection.  For SS, there is over estimation of SSB and underestimation of F, while JABBA shows a strong negative retrospective pattern in F and negative bias in biomass. Although for JABBA  $F:F_{MSY} \le 1$ stock biomass declines below $B_{MSY}$.

The analysis is summarised in Tables \ref{tab:retro-rho} and \ref{tab:retro-re} for Monh's $\rho$ and RMSE respectively. Mohn's $\rho$ has to be in the range $[-0.15,0.2]$ for an assessment to be accepted then (Hurtado-Ferro et al., 2014). All the assessments apart from the base case estimates of F pass the Mohn's $\rho$ test for the retrospective analysis. When the 3 year projection is considered, however, all models other than the ASPM fail. Relative error is harder to interpret. 


The results from the model-free hindcasts are shown in Figures~\ref{fig:hy} and \ref{fig:hy3} for the one and three year ahead predictions respectively. The background colour indicates whether MASE < 1, see Table \ref{tab:tab3} for the MASE values. For the base case and JABBA, prediction skill is poor for CPUE indices 2 and 3; the ASPM also performs poorly for Area 2. Prediction skill further deteriorates for the three step ahead projection, particularly for the base case and JABBA; although for ASPM, CPUE indices 1 and 3 still have good prediction skill. Area 1 is the centre of the stock distribution (move this sentence to methods).

The fits are summarised in Figures~\ref{fig:td} in the form of Taylor diagrams. Although for the one year projection most models appear to have prediction skill, for the three year hindcast the base case and JABBA perform poorly. The model and prediction residuals are summarised in Figure \ref{fig:residuals} over a five year horizon, these show that SS is imprecise and biased and that while JABBA is more precise, it is also biased.

From our analysis the following can be discerned with respect to how well each models with respect to the current status quo in evaluating models, i.e. retrospective analysis. We then present the model free evaluating algorithms with hindcast for one and three year ahead projections. 

The ASPM appears not to show patterns in the projections, CPUE index 2 performs poorly across all models. The difference between the base case is that it uses length composition data, while JABBA does not model the different regions. It therefore, appears that the length compositions add noise and that area effects (JABBA) are important. 
